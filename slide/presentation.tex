\documentclass{beamer}
\usepackage{ctex} % 支持中文
\usepackage{graphicx} % 支持插入图片
\usepackage{booktabs} % 支持更美观的表格

% 主题选择 (可以根据喜好选择其他主题)
% \usetheme{Madrid}
\usetheme{CambridgeUS}
% \usetheme{AnnArbor}
% \usetheme{Boadilla}
% \usetheme{Warsaw}

\title{智能网球检测系统}
\author{黄勖、徐安妮}
\date{\today}

\begin{document}

% 标题页
\begin{frame}
    \titlepage
\end{frame}

% 目录
\begin{frame}{目录}
    \tableofcontents
\end{frame}

%------------------------------------------------
\section{项目概述}
%------------------------------------------------
\begin{frame}{项目概述}
    \begin{itemize}
        \item \textbf{目标}: 开发一个高精度的网球检测系统,主要用于机器人捡球场景。
        \item \textbf{能力}: 在各种光照条件和复杂背景下准确识别网球,处理多个网球。
    \end{itemize}
\end{frame}

\subsection{主要功能}
\begin{frame}{主要功能}
    \begin{itemize}
        \item 单张图片网球检测
        \item 支持多种检测方法(传统图像处理、深度学习)
        \item 适应不同光照和场地条件
    \end{itemize}
\end{frame}

%------------------------------------------------
\section{技术架构}
%------------------------------------------------
\begin{frame}{技术架构}
    系统采用混合检测策略,结合传统计算机视觉和深度学习方法的优点。
    \begin{itemize}
        \item \textbf{传统检测模块} (\texttt{tennis\_detection.py}): 基于颜色和形状。
        \item \textbf{YOLO检测模块} (\texttt{yolo\_singleton.py}): 基于YOLOv5,单例模式,本地缓存。
        \item \textbf{混合检测模块} (\texttt{hybrid\_detection.py}): 结合两者优势,自适应选择。
        \item \textbf{主处理模块} (\texttt{process.py}): 统一接口,错误处理。
    \end{itemize}
\end{frame}

%------------------------------------------------
\section{技术实现细节}
%------------------------------------------------
\subsection{传统检测方法}
\begin{frame}{传统检测方法}
    \begin{enumerate}
        \item \textbf{预处理}: 图像增强、噪声去除。
        \item \textbf{颜色分割}: HSV颜色空间提取特征颜色。
        \item \textbf{形态学操作}: 去除噪点、连接断裂区域。
        \item \textbf{轮廓分析}: 基于面积、圆度等特征过滤。
        \item \textbf{连接网球分离}: 距离变换和分水岭算法。
    \end{enumerate}
\end{frame}

\subsection{YOLO深度学习检测}
\begin{frame}{YOLO深度学习检测}
    基于YOLOv5预训练模型进行优化:
    \begin{itemize}
        \item \textbf{模型选择}: YOLOv5s。
        \item \textbf{单例模式}: 避免重复加载,提高效率。
        \item \textbf{本地缓存}: 模型本地化,避免重复下载。
        \item \textbf{设备优化}: 优先CPU,提高兼容性。
        \item \textbf{错误处理}: 完善的异常捕获。
    \end{itemize}
\end{frame}

\subsection{混合检测策略}
\begin{frame}{混合检测策略}
    组合两种方法的优点:
    \begin{itemize}
        \item \textbf{优先YOLO}: 网络条件允许时首选。
        \item \textbf{自动切换}: YOLO失败时切换到传统方法。
        \item \textbf{结果合并与去重}: 提高准确率。
        \item \textbf{置信度评估}: 选择最终结果。
    \end{itemize}
\end{frame}

%------------------------------------------------
\section{关键问题及解决方案}
%------------------------------------------------
\begin{frame}{关键问题及解决方案}
    \begin{itemize}
        \item \textbf{YOLO模型重复下载}:
            \begin{itemize}
                \item 解决方案: 单例模式、本地缓存、全局状态记录。
            \end{itemize}
        \item \textbf{SSL证书验证错误}:
            \begin{itemize}
                \item 解决方案: 配置SSL上下文、增加超时、错误处理。
            \end{itemize}
        \item \textbf{NumPy类型JSON串行化}:
            \begin{itemize}
                \item 解决方案: 转换为标准Python类型、显式转换。
            \end{itemize}
        \item \textbf{网球连接问题}:
            \begin{itemize}
                \item 解决方案: 分水岭算法、形状后处理、YOLO结果增强。
            \end{itemize}
    \end{itemize}
\end{frame}

%------------------------------------------------
\section{使用指南}
%------------------------------------------------
\begin{frame}[fragile]{使用指南} % 使用 fragile 选项以更好地处理 verbatim 环境
    \textbf{环境准备}:
\begin{verbatim}
pip install -r requirements.txt
\end{verbatim}

    \vspace{0.5cm}
    \textbf{单张图片检测示例}:
\begin{verbatim}
from process import process_img

results = process_img("path/to/image.jpg")
print(results)
\end{verbatim}
\end{frame}

%------------------------------------------------
\section{性能指标}
%------------------------------------------------
\begin{frame}{性能指标}
    \begin{itemize}
        \item \textbf{平均检测时间}: 70-100毫秒/张 (使用缓存模型)
        \item \textbf{准确率}: >90% (标准网球场景)
        \item \textbf{召回率}: >85% (复杂背景场景)
        \item \textbf{误检率}: <5% (针对类似形状物体)
    \end{itemize}
\end{frame}

%------------------------------------------------
\section{项目文档说明}
%------------------------------------------------
\begin{frame}{项目文档说明}
    \begin{itemize}
        \item \texttt{process.py}: 主处理入口 (推荐使用)
        \item \texttt{yolo\_singleton.py}: 优化后的YOLO单例实现
        \item \texttt{src/tennis\_detection.py}: 传统方法检测
        \item \texttt{src/hybrid\_detection.py}: 混合检测算法
        \item \texttt{doc/README.md}: 详细技术文档
        \item \texttt{slide/presentation.tex}: 本演示文稿
    \end{itemize}
\end{frame}

%------------------------------------------------
\section{未来改进方向}
%------------------------------------------------
\begin{frame}{未来改进方向}
    \begin{enumerate}
        \item \textbf{模型优化}: 使用网球专用数据集微调YOLO模型。
        \item \textbf{加速检测}: 探索TensorRT/ONNX等加速方案。
        \item \textbf{多目标跟踪}: 添加目标跟踪功能。
        \item \textbf{距离估计}: 结合相机参数估计网球距离。
        \item \textbf{边缘计算优化}: 针对嵌入式平台优化。
    \end{enumerate}
\end{frame}

%------------------------------------------------
% 致谢/Q&A
%------------------------------------------------
\begin{frame}
    \centering
    {\Huge 谢谢观看!}
    \vfill
    提问与交流
\end{frame}


\end{document}
